\documentclass[11pt,xetex]{beamer}

\usetheme[numbering=none]{metropolis}

\usepackage[swedish]{babel}

\usepackage{subcaption}
\usepackage{gitdags}
\usepackage{hyperref}
\usepackage{minted}
\AtBeginEnvironment{minted}{\fontsize{10}{10}\selectfont}
\usemintedstyle{trac}
\usepackage{perpage}
\MakePerPage{footnote}

\setsansfont{League Spartan}

\definecolor{normalTextColor}{HTML}{303030}
\definecolor{alertedTextColor}{HTML}{EF746F}
\definecolor{exampleTextColor}{HTML}{1693A5}

\setbeamercolor{normal text}{fg=normalTextColor, bg=black!2}
\setbeamercolor{alerted text}{fg=alertedTextColor, bg=black!2}
\setbeamercolor{example text}{fg=exampleTextColor, bg=black!2}

\title{Git till vardags}
\date{\today}
\author{Matthias Nilsson}
%\institute{}

\begin{document}

\maketitle


\section*{Introduktion}

\begin{frame}{Introduktion}
  \Large
  Planen för i dag:

  \begin{itemize}
    \item Commits
    \item Grenar
    \item Samarbete och konflikter
    \item När allt går fel
    \item Tips \& trix
  \end{itemize}
\end{frame}

\begin{frame}{Introduktion}
  \Large
  Några ord om kommandorad vs. GUI.
\end{frame}

\begin{frame}{Introduktion}
  \Large
  Använd det som funkar bäst för dig.
\end{frame}

\section*{Commits}

\begin{frame}[fragile]{Commits}
  \begin{minted}{bash}
    commit c1b8042153cc7ebc459bbf9cad9d61bdc318fb64
    Author: Matthias Nilsson <matthias.nilsson@kb.se>
    Date:   Thu Jan 26 11:18:08 2017 +0100

        Initial commit
  \end{minted}
\end{frame}

\begin{frame}{Commits}
  \Large
  Commits låter oss se vad som ändrats och (förhoppningsvis) varför.
\end{frame}

\begin{frame}{Commits}
  \Large
  \url{http://www.commitlogsfromlastnight.com/}
\end{frame}

\begin{frame}[fragile]{Commits}
  \Large
  Ett bra commit-meddelande förklarar \emph{vad} som gjorts och \emph{varför}.

  \begin{minted}{text}
    En kort (<50 tecken) sammanfattning

    Därefter en längre beskrivande text, om det behövs.
    Försök hålla radlängden till ~72 tecken.

    Imperativ form ("Add", inte "Added") på
    sammanfattningen matchar stilen på de meddelanden
    som Git genererar.
  \end{minted}

  \normalsize
  Se
  \url{http://tbaggery.com/2008/04/19/a-note-about-git-commit-messages.html}
  för mer information.
\end{frame}

\begin{frame}{Commits}
  \Large
  Varje commit som hamnar i \mintinline{text}{master} ska vara komplett.

  \normalsize
  Detta innebär att alla tester går igenom, att koden går att köra utan
  varningar, osv.
\end{frame}

\begin{frame}{Commits}
  \Large
  En bra commit är en logiskt sammanhängande ändring.

  \normalsize
  Orelaterade ändringar ska ligga i andra commits. Håll uppstädning separat
  från logiska ändringar.
\end{frame}

\begin{frame}{Commits}
  \Large
  Gits ''staging area'' gör det här enkelt för oss.
  \pause

  \mintinline{text}{git add --patch} och \mintinline{text}{git add
      --interactive} låter oss välja vad vi vill commit:a.
\end{frame}

\begin{frame}{Commits}
  \Huge
  \alert{OBS}

  \Large
  \mintinline{text}{git commit -m} kan slänga bort information.

  \normalsize
  T.ex. vid en merge med manuellt fixade konflikter.
\end{frame}

\begin{frame}{Commits}
  \Large
  Ibland blir det fel.
  \pause

  \normalsize
  Stavfel i commit-meddelandet, glömt att lägga till en fil, testerna går inte
  igenom, osv.
\end{frame}

\begin{frame}{Commits}
  \Large
  \mintinline{text}{git commit --amend} låter oss ändra på senaste commit:en.

  \normalsize
  (För att ändra på mer finns kommandot \mintinline{text}{git rebase
      --interactive}, men det är överkurs just nu.)
\end{frame}

\begin{frame}{Commits}
  \Large
  Övning 1: Snygga commits
\end{frame}

\begin{frame}{Commits}
  \Large
  Historiken i Git är en riktad acyklisk graf, där varje commit är en nod.

  \small
  (Historiken kan förgrenas, men allt rör sig framåt.)
\end{frame}

\begin{frame}{Commits}
  \Large
  Vi kan använda \mintinline{text}{git log} för att titta på grafen:

  \mintinline{text}{git log --all --decorate --oneline --graph}

  \vspace{1em}

  \normalsize
  (\mintinline{text}{tig}\footnote{\url{https://github.com/jonas/tig}} är värt
  att titta på för den som gillar textbaserade verktyg.)
\end{frame}

\begin{frame}{Commits}
  \Large
  \mintinline{text}{HEAD} är den commit vi står på just nu.

  \small
  \mintinline{text}{HEAD} ändras t.ex. när vi skapar nya commits eller när vi
  byter branch.
\end{frame}

\begin{frame}{Commits}
  \captionsetup{type=table}
  \begin{figure}[t!]
      \begin{subfigure}[b]{0.5\textwidth}
        \centering
        \begin{tikzpicture}
          \gitDAG[grow right sep = 2em]{
            A -- B
          };
          \gitbranch
            [master]
            {master}
            {above=of B}
            {B}
          \gitHEAD
            {above=of master}
            {master}
        \end{tikzpicture}
      \end{subfigure}%
      ~
      \begin{subfigure}[b]{0.5\textwidth}
        \centering
        \begin{tikzpicture}
          \gitDAG[grow right sep = 2em]{
            A -- B -- C
          };
          \gitbranch
            [master]
            {master}
            {above=of C}
            {C}
          \gitHEAD
            {above=of master}
            {master}
        \end{tikzpicture}
      \end{subfigure}
  \end{figure}

  \center
  Före och efter \mintinline{text}{git commit}
\end{frame}

\begin{frame}{Commits}
  \Large
  Förutom att referera till commits med deras commit hash kan vi även referera
  till dem relativt till \mintinline{text}{HEAD}.

  \small
  \mintinline{text}{HEAD~1} är samma sak som att säga ''commit:en som är ett steg innan
  \mintinline{text}{HEAD}''.
\end{frame}

\begin{frame}{Commits}
  \captionsetup{type=table}
  \begin{figure}[t!]
      \begin{subfigure}[b]{\textwidth}
        \centering
        \begin{tikzpicture}
          \gitDAG[grow right sep = 2em]{
            A -- B -- C
          };
          \gitbranch
            [master]
            {master}
            {above=of C}
            {C}
          \gitHEAD
            {above=of master}
            {master}
          \gitremotebranch
            [head1]
            {HEAD\textasciitilde1}
            {above=of B}
            {B}
          \gitremotebranch
            [head2]
            {HEAD\textasciitilde2}
            {above=of A}
            {A}
        \end{tikzpicture}
      \end{subfigure}
  \end{figure}

  \center Relativa commits
\end{frame}

\begin{frame}{Commits}
  \captionsetup{type=table}
  \begin{figure}[t!]
      \begin{subfigure}[b]{0.5\textwidth}
        \centering
        \begin{tikzpicture}
          \gitDAG[grow right sep = 2em]{
            A -- B -- C
          };
          \gitbranch
            {master}
            {above=of C}
            {C}
          \gitHEAD
            {above=of master}
            {master}
        \end{tikzpicture}
      \end{subfigure}%
      ~
      \begin{subfigure}[b]{0.5\textwidth}
        \centering
        \begin{tikzpicture}
          \gitDAG[grow right sep = 2em]{
            A -- B -- {[nodes=unreachable] C}
          };
          \gitbranch
            {master}
            {above=of B}
            {B}
          \gitHEAD
            {above=of master}
            {master}
        \end{tikzpicture}
      \end{subfigure}
  \end{figure}

  \center
  Före och efter \mintinline{text}{git reset HEAD~1}
\end{frame}


\section*{Grenar}

\begin{frame}{Grenar}
  \Large
  Git tillåter (och uppmuntrar) oss att dela upp arbetet i ''grenar''
  (branches).
\end{frame}

\begin{frame}{Grenar}
  \Large
  Varje repo börjar med en \mintinline{text}{master}-gren.
\end{frame}

\begin{frame}{Grenar}
  \Large
  Grenar är billiga att skapa och använda.

  De låter oss:

  \normalsize
  \begin{itemize}
    \item arbeta på flera orelaterade features parallellt
    \item skapa en ny gren för att experimentera utan att riskera att ha
        sönder saker
    \item hantera releaser
  \end{itemize}
\end{frame}

\begin{frame}[fragile]{Grenar}
  \Large
  För att skapa en ny gren:

  \begin{minted}{bash}
    $ git branch my-branch && git checkout my-branch
  \end{minted}

  Genväg:

  \begin{minted}{bash}
    $ git checkout -b my-branch
  \end{minted}
\end{frame}

\begin{frame}[fragile]{Grenar}
  \Large
  För att publicera den nya grenen:

  \begin{minted}{bash}
    $ git push -u origin my-branch
  \end{minted}

  \scriptsize
  (\mintinline{text}{-u} innebär att Git kopplar ihop din lokala gren med den
  på \mintinline{text}{origin}.)
\end{frame}

\begin{frame}{Grenar}
  \Large
  \mintinline{text}{HEAD} följer med när vi byter branch.
\end{frame}

\begin{frame}{Grenar}
  \captionsetup{type=table}
  \begin{subfigure}[t]{0.5\textwidth}
    \centering
    \begin{tikzpicture}
      \gitDAG[grow right sep = 2em]{
        A -- B -- C
      };
      \gitbranch
        {master}
        {above=of B}
        {B}
      \gitbranch
        {my-branch}
        {above=of C}
        {C}
      \gitHEAD
        {above=of my-branch}
        {my-branch}
    \end{tikzpicture}
  \end{subfigure}%
  ~
  \begin{subfigure}[t]{0.5\textwidth}
    \centering
    \begin{tikzpicture}
      \gitDAG[grow right sep = 2em]{
        A -- B -- C
      };
      \gitbranch
        {master}
        {above=of B}
        {B}
      \gitbranch
        {my-branch}
        {above=of C}
        {C}
      \gitHEAD
        {above=of master}
        {master}
    \end{tikzpicture}
  \end{subfigure}

  \center
  Före och efter \mintinline{text}{git checkout master}
\end{frame}

\begin{frame}{Grenar}
  \Large
  För att samla ihop commits från olika grenar använder vi
  \mintinline{text}{git merge} eller \mintinline{text}{git rebase}.
\end{frame}

\begin{frame}{Grenar}
  \captionsetup{type=table}
  \begin{subfigure}[b]{\textwidth}
    \centering
    \begin{tikzpicture}
      \gitDAG[grow right sep = 2em]{
        A -- B -- {
          C,
          D -- E,
        }
      };
      \gitbranch
        [master]
        {master}
        {above=of C}
        {C}
      \gitbranch
        [my-branch]
        {my-branch}
        {below=of E}
        {E}
      \gitHEAD
        {above=of master}
        {master}
    \end{tikzpicture}
  \end{subfigure}
\end{frame}

\begin{frame}{Grenar}
  \captionsetup{type=table}
  \begin{subfigure}[t]{0.5\textwidth}
    \centering
    \begin{tikzpicture}[scale=0.65, every node/.style={transform shape}]
      \gitDAG[grow right sep = 2em]{
        A -- B -- {
          C,
          D -- E,
        } -- F
      };
      \gitbranch
        {master}
        {above=of F}
        {F}
      \gitbranch
        {my-branch}
        {below=of E}
        {E}
      \gitHEAD
        {above=of master}
        {master}
    \end{tikzpicture}
    \mintinline{text}{git merge --no-ff my-branch}
  \end{subfigure}%
  ~
  \begin{subfigure}[t]{0.5\textwidth}
    \centering
    \begin{tikzpicture}[scale=0.65, every node/.style={transform shape}]
      \gitDAG[grow right sep = 2em]{
        A -- B -- {
          D' -- E' -- C,
          D -- E,
        }
      };
      \gitbranch
        {master}
        {above=of C}
        {C}
      \gitbranch
        {my-branch}
        {below=of E}
        {E}
      \gitHEAD
        {above=of master}
        {master}
    \end{tikzpicture}
    \mintinline{text}{git rebase my-branch}
  \end{subfigure}
\end{frame}

\begin{frame}{Grenar}
  \Large
  \begin{exampleblock}{\Huge Tips}
    \begin{itemize}
      \item Använd \mintinline{text}{git merge --no-ff} för att spåra vilken gren
          ändringarna kommer ifrån.
      \item Använd \mintinline{text}{git rebase} när det är irrelevant.
    \end{itemize}
  \end{exampleblock}
\end{frame}

\begin{frame}{Grenar}
  \Large
  Övning 2: Merge och rebase
\end{frame}

\section*{Samarbete och konflikter}

\begin{frame}{Samarbete och konflikter}
  \Large
  Att arbeta tillsammans med andra i samma gren kan vara problematiskt.
\end{frame}

\begin{frame}[fragile]{Samarbete och konflikter}
  \begin{minted}[fontsize=\tiny]{text}
    $ git push
    To git.kb.se:matnil/test.git
     ! [rejected]        master -> master (fetch first)
     error: failed to push some refs to 'git@git.kb.se:matnil/test.git'
     hint: Updates were rejected because the remote contains work that you do
     hint: not have locally. This is usually caused by another repository
     pushing
     hint: to the same ref. You may want to first integrate the remote changes
     hint: (e.g., 'git pull ...') before pushing again.
     hint: See the 'Note about fast-forwards' in 'git push --help' for
     details.
  \end{minted}
\end{frame}

\begin{frame}{Samarbete och konflikter}
  \Large
  \mintinline{text}{git fetch} uppdaterar det lokala repots bild av
  \mintinline{text}{origin}.
\end{frame}

\begin{frame}[fragile]{Samarbete och konflikter}
  \begin{minted}{bash}
  $ git fetch
  Fetching origin
  remote: Counting objects: 3, done.
  remote: Total 3 (delta 0), reused 0 (delta 0)
  Unpacking objects: 100% (3/3), done.
  From git.kb.se:matnil/test
     efe053d..e37aaf7  master     -> origin/master

  $ git status
  On branch master
  Your branch and 'origin/master' have diverged,
  and have 2 and 1 different commits each, respectively.
    (use "git pull" to merge the remote branch into yours)
    nothing to commit, working tree clean
  \end{minted}
\end{frame}

\begin{frame}{Samarbete och konflikter}
  \captionsetup{type=table}
  \begin{subfigure}[t]{\textwidth}
    \centering
    \begin{tikzpicture}[scale=1, every node/.style={transform shape}]
      \gitDAG[grow right sep = 2em]{
        A -- B -- {
          E,
          C -- D,
        }
      };
      \gitremotebranch
        [origmaster]
        {origin/master}
        {above=of E}
        {E}
      \gitbranch
        {master}
        {below=of D}
        {D}
      \gitHEAD
        {below=of master}
        {master}
    \end{tikzpicture}
  \end{subfigure}
\end{frame}

\begin{frame}{Samarbete och konflikter}
  \Large
  \mintinline{text}{git pull} är en genväg för \mintinline{text}{git fetch &&
      git merge}.

  \normalsize
  Ibland är det precis vad vi vill, men andra gånger (som i det här fallet)
  kommer det skapa en onödig merge-commit.
\end{frame}

\begin{frame}{Samarbete och konflikter}
  \Large
  \mintinline{text}{git pull --rebase} låter oss undvika det.
\end{frame}

\begin{frame}{Samarbete och konflikter}
  \captionsetup{type=table}
  \begin{subfigure}[t]{\textwidth}
    \centering
    \begin{tikzpicture}[scale=1, every node/.style={transform shape}]
      \gitDAG[grow right sep = 2em]{
        A -- B -- E -- C -- D
      };
      \gitremotebranch
        [origmaster]
        {origin/master}
        {above=of E}
        {E}
      \gitbranch
        {master}
        {above=of D}
        {D}
      \gitHEAD
        {above=of master}
        {master}
    \end{tikzpicture}
  \end{subfigure}

  \center
  Efter \mintinline{text}{git pull --rebase}
\end{frame}

\begin{frame}{Samarbete och konflikter}
  \begin{alertblock}{\huge Rant}
    \large
    \mintinline{text}{git pull} ska aldrig resultera i en merge-commit.
  \end{alertblock}

  Det tillför inget till historiken.
\end{frame}

\begin{frame}{Samarbete och konflikter}
  \Large
  Om flera personer arbetar i samma repo är konflikter inte ovanligt.

  Konflikter (i koden) är inget att vara rädd för.
\end{frame}

\begin{frame}{Samarbete och konflikter}
  \Large
  Övning 3: Konflikthantering
\end{frame}

\begin{frame}{Samarbete och konflikter}
  \Large
  Kommunikation underlättar vårt arbete.
\end{frame}

\begin{frame}{Samarbete och konflikter}
  \Large
  För snåriga konflikter kan externa verktyg underlätta.

  Ett sådant är \mintinline{text}{meld}\footnote{\url{http://meldmerge.org/}}.
\end{frame}


\section*{När allt går fel}

\begin{frame}{När allt går fel}
  \Large
  Git anstränger sig för att skydda historiken.

  \normalsize
  Det innebär att vi (oftast) behöver anstränga oss för att slänga bort
  information. (T.ex. genom att använda \mintinline{text}{--force}.)
\end{frame}

\begin{frame}{När allt går fel}
  \Large
  \mintinline{text}{git {merge|rebase} --abort}

  \normalsize
  Avbryt och återgå till där du var innan du påbörjade sammanfogningen.
\end{frame}

\begin{frame}{När allt går fel}
  \Large
  \mintinline{text}{git reset --hard origin/master}

  \normalsize
  Släng bort det du har lokalt och börja om med det som ligger remote.
  \pause

  \vspace{1em}
  \alert{OBS}

  \normalsize
  Var noga med att bara köra det här kommandot när det är okej att slänga bort
  lokala ändringar.
\end{frame}

\begin{frame}{När allt går fel}
  \Large
  \mintinline{text}{git reflog}

  \normalsize
  Git loggar varje gång \mintinline{text}{HEAD} ändras och
  \mintinline{text}{git reflog} visar loggen. Vi kan därefter använda
  \mintinline{text}{git checkout} för att gå till ett tidigare steg.

  \scriptsize
  (\mintinline{text}{git checkout} ändrar också \mintinline{text}{HEAD}, så
  Git loggar det också.)
\end{frame}


\section*{Tips \& trix}

\begin{frame}[fragile]{Tips \& trix}
  \Large
  Håll din gren uppdaterad

  \normalsize
  Se till att din gren innehåller senaste \mintinline{text}{master} innan du
  genomför en merge (eller skickar en pull request):

  \begin{minted}{bash}
    $ git checkout -b my-branch
    # jobbajobbajobba
    $ git checkout master && git pull
    $ git checkout my-branch
    $ git rebase master
    # fixa eventuella konflikter
    $ git checkout master && git merge --no-ff my-branch
  \end{minted}
\end{frame}

\begin{frame}[fragile]{Tips \& trix}
  \Large
  Mina top 10 Git-kommandon:

  \begin{minted}{bash}
    $ history | awk '{ print $2, $3 }' | grep "^git" | \
        sort | uniq -c | sort -nr | head -n 10
    855 git status
    514 git diff
    344 git log
    214 git add
    200 git checkout
    179 git push
    176 git commit
    124 git fetch
     94 git stash
     79 git pull
  \end{minted}
\end{frame}

\begin{frame}{Tips \& trix}
  \Large
  \mintinline{text}{git diff --staged}

  \normalsize
  För att kolla att du verkligen commit:ar det du tror.
\end{frame}

\begin{frame}{Tips \& trix}
  \Large
  \mintinline{text}{git tag --annotate}

  \normalsize
  För att markera en release.
\end{frame}

\begin{frame}{Tips \& trix}
  \Large
  \mintinline{text}{git fetch --all}

  \normalsize
  För att hämta hem nya taggar, nya grenar, osv.
\end{frame}

\begin{frame}{Tips \& trix}
  \Large
  \mintinline{text}{git rebase --interactive}

  \normalsize
  För att skriva om historiken.

  \vspace{1em}
  \alert{\Large OBS}

  Det är viktigt att veta när det är okej att skriva om historik och när du
  definitivt inte ska göra det. Det samma gäller \mintinline{text}{git push
      --force}.
\end{frame}

\begin{frame}{Tips \& trix}
  \Large
  \mintinline{text}{git config --global rerere.enabled true}

  Spara lösningar på manuellt fixade konflikter och applicera dem
  automatiskt\footnote{\url{https://git-scm.com/blog/2010/03/08/rerere.html}}.

\end{frame}

\begin{frame}{Tips \& trix}
  \Large
  \mintinline{text}{git cherry-pick}

  \normalsize
  För att plocka in individuella commits i din gren.
\end{frame}

\begin{frame}{Tips \& trix}
  \Large
  \mintinline{text}{git stash}

  \normalsize
  För att (tillfälligt) spara undan ändringar du inte vill commit:a.
\end{frame}

\begin{frame}{Tips \& trix}
  \Large
  \mintinline{text}{git blame}

  \normalsize
  För att ta reda på när en viss rad skrevs och av vem.
\end{frame}

\begin{frame}{Tips \& trix}
  \Large
  Bestäm ett arbetsflöde.

  \normalsize
  Se t.ex. \url{https://www.atlassian.com/git/tutorials/comparing-workflows}.
\end{frame}


\begin{frame}[standout]
  \Huge
  Frågor?
\end{frame}


\section*{Resurser}

\begin{frame}{Resurser}
  \begin{itemize}
    \item Pro Git \\
      \url{https://git-scm.com/book/en/v2}
    \item A Note About Git Commit Messages \\
      \url{http://tbaggery.com/2008/04/19/a-note-about-git-commit-messages.html}
    \item Git for ages 4 and up \\
      \url{https://www.youtube.com/watch?v=1ffBJ4sVUb4}
    \item Git GUI Clients \\
      \url{https://git-scm.com/downloads/guis}
  \end{itemize}
\end{frame}


\begin{frame}[standout]{}
  \begin{center}
    Det här verket är licensierat under Creative Commons
    Attribution-ShareAlike 4.0 International.
    \url{https://creativecommons.org/licenses/by-sa/4.0/}
  \end{center}
\end{frame}


\end{document}

